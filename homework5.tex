\documentclass{article}

\usepackage{fancyhdr}
\usepackage{extramarks}
\usepackage{amsmath}
\usepackage{amsthm}
\usepackage{amsfonts}
\usepackage{tikz}
\usepackage[plain]{algorithm}
\usepackage{algpseudocode}

\usetikzlibrary{automata,positioning}

%
% Basic Document Settings
%

\topmargin=-0.45in
\evensidemargin=0in
\oddsidemargin=0in
\textwidth=6.5in
\textheight=9.0in
\headsep=0.25in

\linespread{1.1}

\pagestyle{fancy}
\lhead{\hmwkAuthorName}
\chead{\hmwkClass\ (\hmwkClassInstructor\ \hmwkClassTime): \hmwkTitle}
\rhead{\firstxmark}
\lfoot{\lastxmark}
\cfoot{\thepage}

\renewcommand\headrulewidth{0.4pt}
\renewcommand\footrulewidth{0.4pt}

\setlength\parindent{0pt}

%
% Create Problem Sections
%

\newcommand{\enterProblemHeader}[1]{
    \nobreak\extramarks{}{Problem \arabic{#1} continued on next page\ldots}\nobreak{}
    \nobreak\extramarks{Problem \arabic{#1} (continued)}{Problem \arabic{#1} continued on next page\ldots}\nobreak{}
}

\newcommand{\exitProblemHeader}[1]{
    \nobreak\extramarks{Problem \arabic{#1} (continued)}{Problem \arabic{#1} continued on next page\ldots}\nobreak{}
    \stepcounter{#1}
    \nobreak\extramarks{Problem \arabic{#1}}{}\nobreak{}
}

\setcounter{secnumdepth}{0}
\newcounter{partCounter}
\newcounter{homeworkProblemCounter}
\setcounter{homeworkProblemCounter}{1}
\nobreak\extramarks{Problem \arabic{homeworkProblemCounter}}{}\nobreak{}

%
% Homework Problem Environment
%
% This environment takes an optional argument. When given, it will adjust the
% problem counter. This is useful for when the problems given for your
% assignment aren't sequential. See the last 3 problems of this template for an
% example.
%
\newenvironment{homeworkProblem}[1][-1]{
    \ifnum#1>0
        \setcounter{homeworkProblemCounter}{#1}
    \fi
    \section{Problem \arabic{homeworkProblemCounter}}
    \setcounter{partCounter}{1}
    \enterProblemHeader{homeworkProblemCounter}
}{
    \exitProblemHeader{homeworkProblemCounter}
}

%
% Homework Details
%   - Title
%   - Due date
%   - Class
%   - Section/Time
%   - Instructor
%   - Author
%

\newcommand{\hmwkTitle}{Homework\ \#5}
\newcommand{\hmwkDueDate}{Oct 24, 2022}
\newcommand{\hmwkClass}{Matrix Calculation}
\newcommand{\hmwkClassTime}{Monday}
\newcommand{\hmwkClassInstructor}{Professor Jun Lai}
\newcommand{\hmwkAuthorName}{\textbf{Shuang Hu}}

%
% Title Page
%

\title{
    \vspace{2in}
    \textmd{\textbf{\hmwkClass:\ \hmwkTitle}}\\
    \normalsize\vspace{0.1in}\small{Due\ on\ \hmwkDueDate\ at 3:10pm}\\
    \vspace{0.1in}\large{\textit{\hmwkClassInstructor\ \hmwkClassTime}}
    \vspace{3in}
}

\author{\hmwkAuthorName}
\date{}

\renewcommand{\part}[1]{\textbf{\large Part \Alph{partCounter}}\stepcounter{partCounter}\\}

%
% Various Helper Commands
%

% Useful for algorithms
\newcommand{\alg}[1]{\textsc{\bfseries \footnotesize #1}}

% For derivatives
\newcommand{\deriv}[1]{\frac{\mathrm{d}}{\mathrm{d}x} (#1)}

% For partial derivatives
\newcommand{\pderiv}[2]{\frac{\partial}{\partial #1} (#2)}

% Integral dx
\newcommand{\dx}{\mathrm{d}x}

% Alias for the Solution section header
\newcommand{\solution}{\textbf{\large Solution}}
\newcommand{\norm}[1]{\|#1\|}
% Probability commands: Expectation, Variance, Covariance, Bias
\newcommand{\E}{\mathrm{E}}
\newcommand{\Var}{\mathrm{Var}}
\newcommand{\Cov}{\mathrm{Cov}}
\newcommand{\Bias}{\mathrm{Bias}}
\newcommand{\supp}{\text{supp}}
\newcommand{\Rn}{\mathbb{R}^{n}}
\newcommand{\dif}{\mathrm{d}}
\newcommand{\avg}[1]{\left\langle #1 \right\rangle}
\newcommand{\difFrac}[2]{\frac{\dif #1}{\dif #2}}
\newcommand{\pdfFrac}[2]{\frac{\partial #1}{\partial #2}}
\newcommand{\OFL}{\mathrm{OFL}}
\newcommand{\UFL}{\mathrm{UFL}}
\newcommand{\fl}{\mathrm{fl}}
\newcommand{\op}{\odot}
\newcommand{\cp}{\cdot}
\newcommand{\Eabs}{E_{\mathrm{abs}}}
\newcommand{\Erel}{E_{\mathrm{rel}}}
\newcommand{\DR}{\mathcal{D}_{\widetilde{LN}}^{n}}
\newcommand{\add}[2]{\sum_{#1=1}^{#2}}
\newcommand{\innerprod}[2]{\left<#1,#2\right>}
\newcommand\tbbint{{-\mkern -16mu\int}}
\newcommand\tbint{{\mathchar '26\mkern -14mu\int}}
\newcommand\dbbint{{-\mkern -19mu\int}}
\newcommand\dbint{{\mathchar '26\mkern -18mu\int}}
\newcommand\bint{
{\mathchoice{\dbint}{\tbint}{\tbint}{\tbint}}
}
\newcommand\bbint{
{\mathchoice{\dbbint}{\tbbint}{\tbbint}{\tbbint}}
}
\begin{document}
\maketitle
\pagebreak
\begin{homeworkProblem}
(P173 Problem4.2.3)

(a) By definition, 
\begin{equation}
\norm{A^{-1}}_{2}=\sup_{x\in\mathbb{R}^{n}}\frac{\norm{A^{-1}x}}{\norm{x}}=\sup_{y\in\mathbb{R}^{n}}\frac{\norm{y}}{\norm{Ay}}
\end{equation}

so:
\begin{equation}
    \begin{aligned}
    \norm{A^{-1}}_{2}^{-1}&=\inf_{\norm{y}=1}\norm{Ay}\\
    &=\inf_{\norm{y}=1}\norm{Ay}\norm{y}\\
    &\ge\inf_{\norm{y}=1}y^{t}Ay=\inf_{\norm{y}=1}y^{t}Ty=\norm{T^{-1}}_{2}^{-1}
    \end{aligned}
\end{equation}
The final equation is derived by the positive definity of matrix $A$. So it's clear that $\norm{A^{-1}}_{2}\le\norm{T^{-1}}_{2}$.

By definition, $T^{-1}-A^{-1}=T^{-1}(A-T)A^{-1}=T^{-1}SA^{-1}$, we can see:
\begin{equation}
    \begin{aligned}
        &x^{t}(T^{-1}-A^{-1})x\\
        =&x^{t}T^{-1}SA^{-1}x\\
        =&(A^{-1}x+(T^{-1}-A^{-1})x)^{t}SA^{-1}x\\
        =&(A^{-1}x+T^{-1}SA^{-1}x)^{t}SA^{-1}x\\
        =&(A^{-1}x)^{t}SA^{-1}x+(SA^{-1}x)^{t}(T^{-1})^{t}SA^{-1}x\\
        \ge&0
    \end{aligned}
\end{equation}
and the final inequality is derived by the anti-symmetrical of $S$, and positive-definity of $T^{-1}$.

(b)By the definition of matrix norm,
\begin{equation}
    \norm{D^{-1}}_{2}=\norm{M^{t}A^{-1}L}_{2}\le\norm{A^{-1}}_{2}\le\norm{T^{-1}}_{2}.
\end{equation}

So $\frac{1}{d_{k}}\le\norm{T^{-1}}_{2}$ $\forall k\Rightarrow d_{k}\ge\frac{1}{\norm{T^{-1}}_{2}}$.
\end{homeworkProblem}
\begin{homeworkProblem}
    Example:
    \begin{equation}
        A=\begin{bmatrix}
        1&-1\\
        1&1\\
        \end{bmatrix}
    \end{equation}
    then:
    \begin{equation}
    \begin{bmatrix}
        x&y\\
    \end{bmatrix}
    A\begin{bmatrix}
        x\\
        y\\
    \end{bmatrix}=x^{2}+y^{2}\ge 0.
\end{equation}
For real vector, the value equals zero if and only if $x=y=0$. For complex vector, we can see:
\begin{equation}
    \begin{bmatrix}
        i&1
    \end{bmatrix}
    \begin{bmatrix}
        1&-1\\
        1&1\\
    \end{bmatrix}
    \begin{bmatrix}
        -i\\
        1\\
    \end{bmatrix}=2-2i\notin\mathbb{R}.
\end{equation}
so $A$ isn't positive definite as $\mathbb{C}^{2\times 2}$.
\end{homeworkProblem}
\begin{homeworkProblem}
    Using theorem 4.2.3, to show $A$ positive definite, it suffices to show $T=\frac{A+A^{t}}{2}$ has positive eigenvalues. As both $A$ and $A^{t}$ are strictly diagonally dominant, we can see that $T$ is strictly diagonally dominant. The diagonal elements of $T$ are positive, i.e. $T$ is positive definite, so the eigenvalues of $T$ are all positive.So $A$ positive definite.
\end{homeworkProblem}
\end{document}