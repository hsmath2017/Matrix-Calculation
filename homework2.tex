\documentclass{article}

\usepackage{fancyhdr}
\usepackage{extramarks}
\usepackage{amsmath}
\usepackage{amsthm}
\usepackage{amsfonts}
\usepackage{tikz}
\usepackage[plain]{algorithm}
\usepackage{algpseudocode}

\usetikzlibrary{automata,positioning}

%
% Basic Document Settings
%

\topmargin=-0.45in
\evensidemargin=0in
\oddsidemargin=0in
\textwidth=6.5in
\textheight=9.0in
\headsep=0.25in

\linespread{1.1}

\pagestyle{fancy}
\lhead{\hmwkAuthorName}
\chead{\hmwkClass\ (\hmwkClassInstructor\ \hmwkClassTime): \hmwkTitle}
\rhead{\firstxmark}
\lfoot{\lastxmark}
\cfoot{\thepage}

\renewcommand\headrulewidth{0.4pt}
\renewcommand\footrulewidth{0.4pt}

\setlength\parindent{0pt}

%
% Create Problem Sections
%

\newcommand{\enterProblemHeader}[1]{
    \nobreak\extramarks{}{Problem \arabic{#1} continued on next page\ldots}\nobreak{}
    \nobreak\extramarks{Problem \arabic{#1} (continued)}{Problem \arabic{#1} continued on next page\ldots}\nobreak{}
}

\newcommand{\exitProblemHeader}[1]{
    \nobreak\extramarks{Problem \arabic{#1} (continued)}{Problem \arabic{#1} continued on next page\ldots}\nobreak{}
    \stepcounter{#1}
    \nobreak\extramarks{Problem \arabic{#1}}{}\nobreak{}
}

\setcounter{secnumdepth}{0}
\newcounter{partCounter}
\newcounter{homeworkProblemCounter}
\setcounter{homeworkProblemCounter}{1}
\nobreak\extramarks{Problem \arabic{homeworkProblemCounter}}{}\nobreak{}

%
% Homework Problem Environment
%
% This environment takes an optional argument. When given, it will adjust the
% problem counter. This is useful for when the problems given for your
% assignment aren't sequential. See the last 3 problems of this template for an
% example.
%
\newenvironment{homeworkProblem}[1][-1]{
    \ifnum#1>0
        \setcounter{homeworkProblemCounter}{#1}
    \fi
    \section{Problem \arabic{homeworkProblemCounter}}
    \setcounter{partCounter}{1}
    \enterProblemHeader{homeworkProblemCounter}
}{
    \exitProblemHeader{homeworkProblemCounter}
}

%
% Homework Details
%   - Title
%   - Due date
%   - Class
%   - Section/Time
%   - Instructor
%   - Author
%

\newcommand{\hmwkTitle}{Homework\ \#2}
\newcommand{\hmwkDueDate}{Sept 26, 2022}
\newcommand{\hmwkClass}{Matrix Calculation}
\newcommand{\hmwkClassTime}{Monday}
\newcommand{\hmwkClassInstructor}{Professor Jun Lai}
\newcommand{\hmwkAuthorName}{\textbf{Shuang Hu}}

%
% Title Page
%

\title{
    \vspace{2in}
    \textmd{\textbf{\hmwkClass:\ \hmwkTitle}}\\
    \normalsize\vspace{0.1in}\small{Due\ on\ \hmwkDueDate\ at 3:10pm}\\
    \vspace{0.1in}\large{\textit{\hmwkClassInstructor\ \hmwkClassTime}}
    \vspace{3in}
}

\author{\hmwkAuthorName}
\date{}

\renewcommand{\part}[1]{\textbf{\large Part \Alph{partCounter}}\stepcounter{partCounter}\\}

%
% Various Helper Commands
%

% Useful for algorithms
\newcommand{\alg}[1]{\textsc{\bfseries \footnotesize #1}}

% For derivatives
\newcommand{\deriv}[1]{\frac{\mathrm{d}}{\mathrm{d}x} (#1)}

% For partial derivatives
\newcommand{\pderiv}[2]{\frac{\partial}{\partial #1} (#2)}

% Integral dx
\newcommand{\dx}{\mathrm{d}x}

% Alias for the Solution section header
\newcommand{\solution}{\textbf{\large Solution}}

% Probability commands: Expectation, Variance, Covariance, Bias
\newcommand{\E}{\mathrm{E}}
\newcommand{\Var}{\mathrm{Var}}
\newcommand{\Cov}{\mathrm{Cov}}
\newcommand{\Bias}{\mathrm{Bias}}
\newcommand{\supp}{\text{supp}}
\newcommand{\Rn}{\mathbb{R}^{n}}
\newcommand{\dif}{\mathrm{d}}
\newcommand{\avg}[1]{\left\langle #1 \right\rangle}
\newcommand{\difFrac}[2]{\frac{\dif #1}{\dif #2}}
\newcommand{\pdfFrac}[2]{\frac{\partial #1}{\partial #2}}
\newcommand{\OFL}{\mathrm{OFL}}
\newcommand{\UFL}{\mathrm{UFL}}
\newcommand{\fl}{\mathrm{fl}}
\newcommand{\op}{\odot}
\newcommand{\cp}{\cdot}
\newcommand{\Eabs}{E_{\mathrm{abs}}}
\newcommand{\Erel}{E_{\mathrm{rel}}}
\newcommand{\DR}{\mathcal{D}_{\widetilde{LN}}^{n}}
\newcommand{\add}[2]{\sum_{#1=1}^{#2}}
\newcommand{\innerprod}[2]{\left<#1,#2\right>}
\newcommand\tbbint{{-\mkern -16mu\int}}
\newcommand\tbint{{\mathchar '26\mkern -14mu\int}}
\newcommand\dbbint{{-\mkern -19mu\int}}
\newcommand\dbint{{\mathchar '26\mkern -18mu\int}}
\newcommand\bint{
{\mathchoice{\dbint}{\tbint}{\tbint}{\tbint}}
}
\newcommand\bbint{
{\mathchoice{\dbbint}{\tbbint}{\tbbint}{\tbbint}}
}
\begin{document}
\maketitle
\pagebreak
\begin{homeworkProblem}
    (Page 80, Problem 2.4.2)
    \begin{proof}
    By SVD decomposition, there exists orthogonal matrix $U,V$ such that
    \begin{equation}
        \label{eq:SVD}
        U^{t}AV=\begin{bmatrix}
            D&O\\
            O&O\\
        \end{bmatrix}
        , D=\text{diag}(\sigma_{1},\cdots,\sigma_{r}).
    \end{equation}
    WLOG, we set $\sigma_{1}\ge\sigma_{2}\ge\cdots\ge\sigma_{r}$, and assume that $U=[y_{1},\cdots,y_{m}]$, $V=[x_{1},\cdots,x_{n}]$, it means that:
    \begin{equation}
        y_{i}^{t}Ax_{j}=
        \left\{
            \begin{aligned}
                &\sigma_{i}\,(i=j,i,j\le r)\\
                &0\,(otherwise).
            \end{aligned}
        \right.
    \end{equation}
    where $\|y_{i}\|=\|x_{i}\|=1$. As $\{y_{i}\}$, $\{x_{j}\}$ are orthonormal basis, respectively, we can write:
    \begin{equation}
        \begin{aligned}
            x&=\sum_{i=1}^{n}\beta_{i}x_{i}\\
            y&=\sum_{i=1}^{m}\alpha_{i}y_{i}\\
        \end{aligned}
    \end{equation}
    Then:
    \begin{equation}
        \begin{aligned}
            y^{t}Ax&=\sum\alpha_{i}\beta_{i}\sigma_{i}\\
            &\le\sigma_{\max}\sum|\alpha_{i}\beta_{i}|\\
            &\le\sigma_{\max}\sqrt{\sum\alpha_{i}^{2}\sum\beta_{i}^{2}}\\
            &=\sigma_{\max}\|x\|_{2}\|y\|_{2}.\\
        \end{aligned}
    \end{equation}
    It means that 
    \begin{equation}
        \sigma_{\max}\ge\max\frac{y^{t}Ax}{\|x\|_{2}\|y\|_{2}}.
    \end{equation}
    On the other hand, set $x=x_{1}$, $y=y_{1}$, we can see $\frac{y^{t}Ax}{\|x\|_{2}\|y\|_{2}}=\sigma_{1}=\sigma_{\max}$. Q.E.D.
    \end{proof}
\end{homeworkProblem}
\begin{homeworkProblem}
    (Page 80, Problem 2.4.6)

    By corollary 2.4.7, $A=\sigma_{1}u_{1}v_{1}^{t}+\sigma_{2}u_{2}v_{2}^{t}$. In one hand, set $B=\sigma_{1}u_{1}v_{1}^{t}$, we can get $\|A-B\|_{F}=\sigma_{2}$.

    In the other hand, assume $r(B)=1$, then $\dim\ker B=1$, assume that $Bx=0$. Expand the vector $x$ to an orthonormal basis for $\mathbb{R}^{2}$ as $\{x,y\}$, then:
    \begin{equation}
        \begin{aligned}
            \|A-B\|_{F}^{2}&\ge\|(A-B)x\|_{2}^{2}\\
            &=\|Ax\|_{2}^{2}\\
            &=\sigma_{1}^{2}(v_{1}^{t}x)^{2}+\sigma_{2}^{2}(v_{2}^{t}x)^{2}\\
            &\ge\sigma_{2}^{2}.
        \end{aligned}
    \end{equation}

    So the nearest rank-1 matrix $B=\sigma_{1}u_{1}v_{1}^{t}$.
\end{homeworkProblem}
\begin{homeworkProblem}
    (Page 80, Problem 2.4.7)

    \begin{proof}
        By corollary 2.4.3, we can see:
        \begin{equation}
            \begin{aligned}
                \|A\|_{F}&=\sqrt{\sum_{i=1}^{r}\sigma_{i}^{2}}\\
                \|A\|_{2}&=\sigma_{i}.
            \end{aligned}
        \end{equation}
    As $\sum_{i=1}^{r}\sigma_{i}^{2}\le r\sigma_{1}^{2}$, it means that $\|A\|_{F}\le\sqrt{r}\sigma_{1}=\sqrt{r}\|A\|_{2}$.
    \end{proof}
\end{homeworkProblem}
\end{document}