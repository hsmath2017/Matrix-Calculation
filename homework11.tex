\documentclass{article}

\usepackage{fancyhdr}
\usepackage{extramarks}
\usepackage{amsmath}
\usepackage{amsthm}
\usepackage{amsfonts}
\usepackage{tikz}
\usepackage[plain]{algorithm}
\usepackage{algpseudocode}

\usetikzlibrary{automata,positioning}

%
% Basic Document Settings
%

\topmargin=-0.45in
\evensidemargin=0in
\oddsidemargin=0in
\textwidth=6.5in
\textheight=9.0in
\headsep=0.25in

\linespread{1.1}

\pagestyle{fancy}
\lhead{\hmwkAuthorName}
\chead{\hmwkClass\ (\hmwkClassInstructor\ \hmwkClassTime): \hmwkTitle}
\rhead{\firstxmark}
\lfoot{\lastxmark}
\cfoot{\thepage}

\renewcommand\headrulewidth{0.4pt}
\renewcommand\footrulewidth{0.4pt}

\setlength\parindent{0pt}

%
% Create Problem Sections
%

\newcommand{\enterProblemHeader}[1]{
    \nobreak\extramarks{}{Problem \arabic{#1} continued on next page\ldots}\nobreak{}
    \nobreak\extramarks{Problem \arabic{#1} (continued)}{Problem \arabic{#1} continued on next page\ldots}\nobreak{}
}

\newcommand{\exitProblemHeader}[1]{
    \nobreak\extramarks{Problem \arabic{#1} (continued)}{Problem \arabic{#1} continued on next page\ldots}\nobreak{}
    \stepcounter{#1}
    \nobreak\extramarks{Problem \arabic{#1}}{}\nobreak{}
}

\setcounter{secnumdepth}{0}
\newcounter{partCounter}
\newcounter{homeworkProblemCounter}
\setcounter{homeworkProblemCounter}{1}
\nobreak\extramarks{Problem \arabic{homeworkProblemCounter}}{}\nobreak{}

%
% Homework Problem Environment
%
% This environment takes an optional argument. When given, it will adjust the
% problem counter. This is useful for when the problems given for your
% assignment aren't sequential. See the last 3 problems of this template for an
% example.
%
\newenvironment{homeworkProblem}[1][-1]{
    \ifnum#1>0
        \setcounter{homeworkProblemCounter}{#1}
    \fi
    \section{Problem \arabic{homeworkProblemCounter}}
    \setcounter{partCounter}{1}
    \enterProblemHeader{homeworkProblemCounter}
}{
    \exitProblemHeader{homeworkProblemCounter}
}

%
% Homework Details
%   - Title
%   - Due date
%   - Class
%   - Section/Time
%   - Instructor
%   - Author
%

\newcommand{\hmwkTitle}{Homework\ \#11}
\newcommand{\hmwkDueDate}{Dec 12, 2022}
\newcommand{\hmwkClass}{Matrix Calculation}
\newcommand{\hmwkClassTime}{Monday}
\newcommand{\hmwkClassInstructor}{Professor Jun Lai}
\newcommand{\hmwkAuthorName}{\textbf{Shuang Hu}}

%
% Title Page
%

\title{
    \vspace{2in}
    \textmd{\textbf{\hmwkClass:\ \hmwkTitle}}\\
    \normalsize\vspace{0.1in}\small{Due\ on\ \hmwkDueDate\ at 3:10pm}\\
    \vspace{0.1in}\large{\textit{\hmwkClassInstructor\ \hmwkClassTime}}
    \vspace{3in}
}

\author{\hmwkAuthorName}
\date{}

\renewcommand{\part}[1]{\textbf{\large Part \Alph{partCounter}}\stepcounter{partCounter}\\}

%
% Various Helper Commands
%

% Useful for algorithms
\newcommand{\alg}[1]{\textsc{\bfseries \footnotesize #1}}

% For derivatives
\newcommand{\deriv}[1]{\frac{\mathrm{d}}{\mathrm{d}x} (#1)}

% For partial derivatives
\newcommand{\pderiv}[2]{\frac{\partial}{\partial #1} (#2)}

% Integral dx
\newcommand{\dx}{\mathrm{d}x}

% Alias for the Solution section header
\newcommand{\solution}{\textbf{\large Solution}}
\newcommand{\norm}[1]{\|#1\|}
% Probability commands: Expectation, Variance, Covariance, Bias
\newcommand{\E}{\mathrm{E}}
\newcommand{\Var}{\mathrm{Var}}
\newcommand{\Cov}{\mathrm{Cov}}
\newcommand{\Bias}{\mathrm{Bias}}
\newcommand{\supp}{\text{supp}}
\newcommand{\Rn}{\mathbb{R}^{n}}
\newcommand{\dif}{\mathrm{d}}
\newcommand{\avg}[1]{\left\langle #1 \right\rangle}
\newcommand{\difFrac}[2]{\frac{\dif #1}{\dif #2}}
\newcommand{\pdfFrac}[2]{\frac{\partial #1}{\partial #2}}
\newcommand{\OFL}{\mathrm{OFL}}
\newcommand{\UFL}{\mathrm{UFL}}
\newcommand{\fl}{\mathrm{fl}}
\newcommand{\op}{\odot}
\newcommand{\cp}{\cdot}
\newcommand{\Eabs}{E_{\mathrm{abs}}}
\newcommand{\Erel}{E_{\mathrm{rel}}}
\newcommand{\DR}{\mathcal{D}_{\widetilde{LN}}^{n}}
\newcommand{\add}[2]{\sum_{#1=1}^{#2}}
\newcommand{\innerprod}[2]{\left<#1,#2\right>}
\newcommand\tbbint{{-\mkern -16mu\int}}
\newcommand\tbint{{\mathchar '26\mkern -14mu\int}}
\newcommand\dbbint{{-\mkern -19mu\int}}
\newcommand\dbint{{\mathchar '26\mkern -18mu\int}}
\newcommand\bint{
{\mathchoice{\dbint}{\tbint}{\tbint}{\tbint}}
}
\newcommand\bbint{
{\mathchoice{\dbbint}{\tbbint}{\tbbint}{\tbbint}}
}
\begin{document}
\maketitle
\pagebreak
\begin{homeworkProblem}
    (P383 Problem7.4.4)

    Denote Hessenberg matrix 
    \begin{equation}
        H=\begin{bmatrix}
            h_{11}&h_{12}&\cdots&h_{1n}\\
            h_{21}&h_{22}&\cdots&h_{2n}\\
            &\ddots&\ddots&\vdots&\\
            &&h_{n-1,n}&h_{nn}&\\
        \end{bmatrix}
    \end{equation}
    and the diagonal matrix 
    \begin{equation}
        D=\text{diag}\{d_{1},d_{2},\cdots,d_{n}\}.
    \end{equation}
    As the subdiagonal elements of $D^{-1}HD$ are all 1, we can see that:
    \begin{equation}
        \frac{d_{i+1}}{d_{i}}=h_{i+1,i},i\in[1,n-1].
    \end{equation}
    It derives the value of matrix $D$. Its condition number:
    \begin{equation}
        \kappa_{2}(D)=\frac{\max\prod_{i}|h_{i,i-1}|}{\min\prod_{i}|h_{i,i-1}|}.
    \end{equation}
\end{homeworkProblem}
\begin{homeworkProblem}
    (P383 Problem7.4.5)

    Assume the eigenvector related to $\lambda$ is $X_{1}+iX_{2}$, we can see:
    \begin{equation}
        (W+iY)(X_{1}+iX_{2})=\lambda(X_{1}+iX_{2}).
    \end{equation}
    i.e.
    \begin{equation}
        (WX_{1}-YX_{2})+i(YX_{1}+WX_{2})=\lambda X_{1}+i\lambda X_{2}.
    \end{equation}
    Then:
    \begin{equation}
        \left\{
            \begin{aligned}
                WX_{1}-YX_{2}&=\lambda X_{1},\\
                YX_{1}+WX_{2}&=\lambda X_{2}.\\
            \end{aligned}
        \right.
    \end{equation}
    i.e.
    \begin{equation}
        \begin{bmatrix}
            W&-Y\\
            Y&W\\
        \end{bmatrix}
        \begin{bmatrix}
            X_{1}\\
            X_{2}\\
        \end{bmatrix}
        =\lambda\begin{bmatrix}
            X_{1}\\
            X_{2}\\
        \end{bmatrix}.
    \end{equation}
    So $\lambda$ is an eigenvalue of $B$, with the eigenvector $\begin{bmatrix}
        X_{1}\\
        X_{2}\\
    \end{bmatrix}$.
\end{homeworkProblem}
\begin{homeworkProblem}
    (Page 392, Problem7.5.5)

    For $j=2$, consider the induction basis:
    \begin{equation}
        \begin{aligned}
        U_{1}U_{2}R_{2}R_{1}&=U_{1}(H_{2}-\mu_{2}I)R_{1}\\
        &=U_{1}H_{2}R_{1}-\mu_{2}(H-\mu_{1}I)\\
        &=U_{1}(R_{1}U_{1}+\mu_{1}I)R_{1}-\mu_{2}(H-\mu_{1}I)\\
        &=(H-\mu_{1}I)^{2}+\mu_{1}(H-\mu_{1}I)-\mu_{2}(H-\mu_{1}I)\\
        &=H^{2}-(\mu_{1}+\mu_{2})H+\mu_{1}\mu_{2}I\\
        &=(H-\mu_{1}I)(H-\mu_{2}I).
        \end{aligned}
    \end{equation}
    Assume the result holds for $j=n$, i.e. 
    \begin{equation}
        U_{1}\cdots U_{n}R_{n}\cdots R_{1}=(H-\mu_{1}I)\cdots(H-\mu_{n}I).
    \end{equation}
    Then:
    \begin{equation}
        U_{1}\cdots U_{n+1}R_{n+1}\cdots R_{1}=U_{1}\cdots U_{n}H_{n+1}R_{n}\cdots R_{1}-\mu_{n+1}(H-\mu_{1}I)\cdots (H-\mu_{n}I).
    \end{equation}
    As:
    \begin{equation}
        H_{j+1}R_{j}=(R_{j}U_{j}+\mu_{j}I)R_{j}=R_{j}(\mu_{j}I+U_{j}R_{j})=R_{j}H_{j}.
    \end{equation}
    It means that:
    \begin{equation}
        U_{1}\cdots U_{n}H_{n+1}R_{n}\cdots R_{1}=U_{1}\cdots U_{n}R_{n}\cdots R_{1}H.
    \end{equation}
    So the result holds for $j=n+1$.
\end{homeworkProblem}
\end{document}