\documentclass{article}

\usepackage{fancyhdr}
\usepackage{extramarks}
\usepackage{amsmath}
\usepackage{amsthm}
\usepackage{amsfonts}
\usepackage{tikz}
\usepackage[plain]{algorithm}
\usepackage{algpseudocode}

\usetikzlibrary{automata,positioning}

%
% Basic Document Settings
%

\topmargin=-0.45in
\evensidemargin=0in
\oddsidemargin=0in
\textwidth=6.5in
\textheight=9.0in
\headsep=0.25in

\linespread{1.1}

\pagestyle{fancy}
\lhead{\hmwkAuthorName}
\chead{\hmwkClass\ (\hmwkClassInstructor\ \hmwkClassTime): \hmwkTitle}
\rhead{\firstxmark}
\lfoot{\lastxmark}
\cfoot{\thepage}

\renewcommand\headrulewidth{0.4pt}
\renewcommand\footrulewidth{0.4pt}

\setlength\parindent{0pt}

%
% Create Problem Sections
%

\newcommand{\enterProblemHeader}[1]{
    \nobreak\extramarks{}{Problem \arabic{#1} continued on next page\ldots}\nobreak{}
    \nobreak\extramarks{Problem \arabic{#1} (continued)}{Problem \arabic{#1} continued on next page\ldots}\nobreak{}
}

\newcommand{\exitProblemHeader}[1]{
    \nobreak\extramarks{Problem \arabic{#1} (continued)}{Problem \arabic{#1} continued on next page\ldots}\nobreak{}
    \stepcounter{#1}
    \nobreak\extramarks{Problem \arabic{#1}}{}\nobreak{}
}

\setcounter{secnumdepth}{0}
\newcounter{partCounter}
\newcounter{homeworkProblemCounter}
\setcounter{homeworkProblemCounter}{1}
\nobreak\extramarks{Problem \arabic{homeworkProblemCounter}}{}\nobreak{}

%
% Homework Problem Environment
%
% This environment takes an optional argument. When given, it will adjust the
% problem counter. This is useful for when the problems given for your
% assignment aren't sequential. See the last 3 problems of this template for an
% example.
%
\newenvironment{homeworkProblem}[1][-1]{
    \ifnum#1>0
        \setcounter{homeworkProblemCounter}{#1}
    \fi
    \section{Problem \arabic{homeworkProblemCounter}}
    \setcounter{partCounter}{1}
    \enterProblemHeader{homeworkProblemCounter}
}{
    \exitProblemHeader{homeworkProblemCounter}
}

%
% Homework Details
%   - Title
%   - Due date
%   - Class
%   - Section/Time
%   - Instructor
%   - Author
%

\newcommand{\hmwkTitle}{Homework\ \#4}
\newcommand{\hmwkDueDate}{Oct 17, 2022}
\newcommand{\hmwkClass}{Matrix Calculation}
\newcommand{\hmwkClassTime}{Monday}
\newcommand{\hmwkClassInstructor}{Professor Jun Lai}
\newcommand{\hmwkAuthorName}{\textbf{Shuang Hu}}

%
% Title Page
%

\title{
    \vspace{2in}
    \textmd{\textbf{\hmwkClass:\ \hmwkTitle}}\\
    \normalsize\vspace{0.1in}\small{Due\ on\ \hmwkDueDate\ at 3:10pm}\\
    \vspace{0.1in}\large{\textit{\hmwkClassInstructor\ \hmwkClassTime}}
    \vspace{3in}
}

\author{\hmwkAuthorName}
\date{}

\renewcommand{\part}[1]{\textbf{\large Part \Alph{partCounter}}\stepcounter{partCounter}\\}

%
% Various Helper Commands
%

% Useful for algorithms
\newcommand{\alg}[1]{\textsc{\bfseries \footnotesize #1}}

% For derivatives
\newcommand{\deriv}[1]{\frac{\mathrm{d}}{\mathrm{d}x} (#1)}

% For partial derivatives
\newcommand{\pderiv}[2]{\frac{\partial}{\partial #1} (#2)}

% Integral dx
\newcommand{\dx}{\mathrm{d}x}

% Alias for the Solution section header
\newcommand{\solution}{\textbf{\large Solution}}

% Probability commands: Expectation, Variance, Covariance, Bias
\newcommand{\E}{\mathrm{E}}
\newcommand{\Var}{\mathrm{Var}}
\newcommand{\Cov}{\mathrm{Cov}}
\newcommand{\Bias}{\mathrm{Bias}}
\newcommand{\supp}{\text{supp}}
\newcommand{\Rn}{\mathbb{R}^{n}}
\newcommand{\dif}{\mathrm{d}}
\newcommand{\avg}[1]{\left\langle #1 \right\rangle}
\newcommand{\difFrac}[2]{\frac{\dif #1}{\dif #2}}
\newcommand{\pdfFrac}[2]{\frac{\partial #1}{\partial #2}}
\newcommand{\OFL}{\mathrm{OFL}}
\newcommand{\UFL}{\mathrm{UFL}}
\newcommand{\fl}{\mathrm{fl}}
\newcommand{\op}{\odot}
\newcommand{\cp}{\cdot}
\newcommand{\Eabs}{E_{\mathrm{abs}}}
\newcommand{\Erel}{E_{\mathrm{rel}}}
\newcommand{\DR}{\mathcal{D}_{\widetilde{LN}}^{n}}
\newcommand{\add}[2]{\sum_{#1=1}^{#2}}
\newcommand{\innerprod}[2]{\left<#1,#2\right>}
\newcommand\tbbint{{-\mkern -16mu\int}}
\newcommand\tbint{{\mathchar '26\mkern -14mu\int}}
\newcommand\dbbint{{-\mkern -19mu\int}}
\newcommand\dbint{{\mathchar '26\mkern -18mu\int}}
\newcommand\bint{
{\mathchoice{\dbint}{\tbint}{\tbint}{\tbint}}
}
\newcommand\bbint{
{\mathchoice{\dbbint}{\tbbint}{\tbbint}{\tbbint}}
}
\begin{document}
\maketitle
\pagebreak
\begin{homeworkProblem}
    (P136 Problem1)

    By the definition of matrix production, we can see:
    \begin{equation}
        \begin{bmatrix}
            a_{1}^{t}\\
            a_{2}^{t}\\
            \vdots\\
            a_{n}^{t}\\
        \end{bmatrix}
        =\begin{bmatrix}
            l_{11}& & & &\\
            l_{21}&l_{22}& & &\\
            \vdots&\vdots&\ddots& &\\
            l_{n1}&l_{n1}&\cdots&l_{nn}&\\
        \end{bmatrix}
        \begin{bmatrix}
            u_{1}^{t}\\
            u_{2}^{t}\\
            \vdots\\
            u_{n}^{t}\\
        \end{bmatrix}
    \end{equation}
    As $l_{ii}=1$, we can see that 
    \begin{equation}
        u_{i}^{t}=a_{i}^{t}-\sum_{j=1}^{i-1}l_{ij}u_{j}^{t}.
    \end{equation}
    To prove $\|U\|_{\infty}\le 2^{n-1}\|A\|_{\infty}$, I use induction on the row order $k\le n$. First, I claim that 
    \begin{equation}
        \|u_{k}^{t}\|_{1}\le 2^{k-1}\|A\|_{\infty}.
    \end{equation}
    For $k=1$, we can see $\|u_{1}^{t}\|_{1}=\|a_{1}^{t}\|_{1}\le\|A\|_{\infty}$.

    Assume the result is true for $k\le k_{0}$, for $k=k_{0}+1$, we can see:
    \begin{equation}
        \|u_{k_{0}+1}\|_{1}\le\|a_{k_{0}+1}\|_{1}+\sum_{j=1}^{k_{0}}|l_{ij}|\|u_{j}\|_{1}\le\|A\|_{\infty}+\sum_{j=1}^{k_{0}}2^{j-1}\|A\|_{\infty}=2^{k_{0}}\|A\|_{\infty}
    \end{equation}
    This equation shows that $(3)$ is correct. And $(3)$ suggests that $\|U\|_{\infty}\le 2^{n-1}\|A\|_{\infty}$, as $\|U\|_{\infty}=\max{\|u_{i}\|_{1}}$.
\end{homeworkProblem}
\begin{homeworkProblem}
    (P158 Problem1)

    Assume the matrix $A$ is singular, it suggests that $\exists x\neq 0$ such that $Ax=0$. Write $x=\begin{bmatrix}
        x_{1}\\
        x_{2}\\
        \vdots\\
        x_{n}\\
    \end{bmatrix}$, assume $|x_{i}|=\max_{j}\{|x_{j}|\}$, consider the ith row of matrix A, we can see:
    \begin{equation}
        \sum a_{ij}x_{j}=0.
    \end{equation}
    $(5)$ suggests that 
    \begin{equation}
        a_{ii}=-\sum_{j\neq i}a_{ij}\frac{x_{j}}{x_{i}}.
    \end{equation}
    Then:
    \begin{equation}
        |a_{ii}|\le\sum_{j\neq i}|a_{ij}|.
    \end{equation}
    Contradict to the strict diagonal advantage. So matrix $A$ must be nonsingular.
\end{homeworkProblem}
\begin{homeworkProblem}
    (P158 Problem2)

    $A$ is row diagonally dominant means that $A^{t}$ is column diagonally dominant, then use theorem 4.1.2, we can see that $\|(A^{t})^{-1}\|_{1}\le\frac{1}{\delta}$. It's clear that $A^{t}_{1}=\|A\|_{\infty}$, so $\|A^{-1}\|_{\infty}\le\frac{1}{\delta}$.
\end{homeworkProblem}
\end{document}