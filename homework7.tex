\documentclass{article}

\usepackage{fancyhdr}
\usepackage{extramarks}
\usepackage{amsmath}
\usepackage{amsthm}
\usepackage{amsfonts}
\usepackage{tikz}
\usepackage[plain]{algorithm}
\usepackage{algpseudocode}

\usetikzlibrary{automata,positioning}

%
% Basic Document Settings
%

\topmargin=-0.45in
\evensidemargin=0in
\oddsidemargin=0in
\textwidth=6.5in
\textheight=9.0in
\headsep=0.25in

\linespread{1.1}

\pagestyle{fancy}
\lhead{\hmwkAuthorName}
\chead{\hmwkClass\ (\hmwkClassInstructor\ \hmwkClassTime): \hmwkTitle}
\rhead{\firstxmark}
\lfoot{\lastxmark}
\cfoot{\thepage}

\renewcommand\headrulewidth{0.4pt}
\renewcommand\footrulewidth{0.4pt}

\setlength\parindent{0pt}

%
% Create Problem Sections
%

\newcommand{\enterProblemHeader}[1]{
    \nobreak\extramarks{}{Problem \arabic{#1} continued on next page\ldots}\nobreak{}
    \nobreak\extramarks{Problem \arabic{#1} (continued)}{Problem \arabic{#1} continued on next page\ldots}\nobreak{}
}

\newcommand{\exitProblemHeader}[1]{
    \nobreak\extramarks{Problem \arabic{#1} (continued)}{Problem \arabic{#1} continued on next page\ldots}\nobreak{}
    \stepcounter{#1}
    \nobreak\extramarks{Problem \arabic{#1}}{}\nobreak{}
}

\setcounter{secnumdepth}{0}
\newcounter{partCounter}
\newcounter{homeworkProblemCounter}
\setcounter{homeworkProblemCounter}{1}
\nobreak\extramarks{Problem \arabic{homeworkProblemCounter}}{}\nobreak{}

%
% Homework Problem Environment
%
% This environment takes an optional argument. When given, it will adjust the
% problem counter. This is useful for when the problems given for your
% assignment aren't sequential. See the last 3 problems of this template for an
% example.
%
\newenvironment{homeworkProblem}[1][-1]{
    \ifnum#1>0
        \setcounter{homeworkProblemCounter}{#1}
    \fi
    \section{Problem \arabic{homeworkProblemCounter}}
    \setcounter{partCounter}{1}
    \enterProblemHeader{homeworkProblemCounter}
}{
    \exitProblemHeader{homeworkProblemCounter}
}

%
% Homework Details
%   - Title
%   - Due date
%   - Class
%   - Section/Time
%   - Instructor
%   - Author
%

\newcommand{\hmwkTitle}{Homework\ \#7}
\newcommand{\hmwkDueDate}{Nov 7, 2022}
\newcommand{\hmwkClass}{Matrix Calculation}
\newcommand{\hmwkClassTime}{Monday}
\newcommand{\hmwkClassInstructor}{Professor Jun Lai}
\newcommand{\hmwkAuthorName}{\textbf{Shuang Hu}}

%
% Title Page
%

\title{
    \vspace{2in}
    \textmd{\textbf{\hmwkClass:\ \hmwkTitle}}\\
    \normalsize\vspace{0.1in}\small{Due\ on\ \hmwkDueDate\ at 3:10pm}\\
    \vspace{0.1in}\large{\textit{\hmwkClassInstructor\ \hmwkClassTime}}
    \vspace{3in}
}

\author{\hmwkAuthorName}
\date{}

\renewcommand{\part}[1]{\textbf{\large Part \Alph{partCounter}}\stepcounter{partCounter}\\}

%
% Various Helper Commands
%

% Useful for algorithms
\newcommand{\alg}[1]{\textsc{\bfseries \footnotesize #1}}

% For derivatives
\newcommand{\deriv}[1]{\frac{\mathrm{d}}{\mathrm{d}x} (#1)}

% For partial derivatives
\newcommand{\pderiv}[2]{\frac{\partial}{\partial #1} (#2)}

% Integral dx
\newcommand{\dx}{\mathrm{d}x}

% Alias for the Solution section header
\newcommand{\solution}{\textbf{\large Solution}}
\newcommand{\norm}[1]{\|#1\|}
% Probability commands: Expectation, Variance, Covariance, Bias
\newcommand{\E}{\mathrm{E}}
\newcommand{\Var}{\mathrm{Var}}
\newcommand{\Cov}{\mathrm{Cov}}
\newcommand{\Bias}{\mathrm{Bias}}
\newcommand{\supp}{\text{supp}}
\newcommand{\Rn}{\mathbb{R}^{n}}
\newcommand{\dif}{\mathrm{d}}
\newcommand{\avg}[1]{\left\langle #1 \right\rangle}
\newcommand{\difFrac}[2]{\frac{\dif #1}{\dif #2}}
\newcommand{\pdfFrac}[2]{\frac{\partial #1}{\partial #2}}
\newcommand{\OFL}{\mathrm{OFL}}
\newcommand{\UFL}{\mathrm{UFL}}
\newcommand{\fl}{\mathrm{fl}}
\newcommand{\op}{\odot}
\newcommand{\cp}{\cdot}
\newcommand{\Eabs}{E_{\mathrm{abs}}}
\newcommand{\Erel}{E_{\mathrm{rel}}}
\newcommand{\DR}{\mathcal{D}_{\widetilde{LN}}^{n}}
\newcommand{\add}[2]{\sum_{#1=1}^{#2}}
\newcommand{\innerprod}[2]{\left<#1,#2\right>}
\newcommand\tbbint{{-\mkern -16mu\int}}
\newcommand\tbint{{\mathchar '26\mkern -14mu\int}}
\newcommand\dbbint{{-\mkern -19mu\int}}
\newcommand\dbint{{\mathchar '26\mkern -18mu\int}}
\newcommand\bint{
{\mathchoice{\dbint}{\tbint}{\tbint}{\tbint}}
}
\newcommand\bbint{
{\mathchoice{\dbbint}{\tbbint}{\tbbint}{\tbbint}}
}
\begin{document}
\maketitle
\pagebreak
\begin{homeworkProblem}
(Page 232,Problem 4.8.8)

For a point $P$ whose discretrize stencil is totally inside the domain $\Omega$, the equation below is true:
\begin{equation}
    \frac{u(N)+u(S)+u(E)+u(W)-4u(P)}{h^{2}}=F(P).
\end{equation}
Assume the identifier of grid is the same as Figure 4.8.1, then for boundary cells near the left-side, a.g. $u_{n_{1}+1}$, we can write the following discreterize equation:
\begin{equation}
    4u_{n_{1}+1}-u_{1}-u_{2}-u_{2n_{1}+1}=h^{2}F_{n_{1}+1}.
\end{equation}
while the equation is derived by $(1)$ and boundary condition $u|_{x=0}=0$.

For the bottom side of the problem domain, we can use totally the same method to discuss this problem. Then, for the Neumann-Boundary side points, such as $u_{2}$, it's a good idea to use so-called \textbf{Ghost Cell} to handle it. But we can also write the discreterize function like this:
\begin{equation}
    \left\{
        \begin{aligned}
            \frac{u_{1}+u_{3}+u_{n_{1}+2}+u_{G}-4u_{2}}{h^{2}}&=F(2)\\
            \frac{u_{G}-u_{n_{1}+2}}{2h}&=0\\
        \end{aligned}
    \right.
\end{equation}
This discreterized equation is derived by the homogenious Neumann boundary condition. $(1)$, $(2)$ and $(3)$ makes the coefficient matrix $A$. And in the same time, the right-hand vector $b$ can be derived by the function $F$, which is:
\begin{equation}
    b=h^{2}\begin{bmatrix}
        F(1)\\
        \vdots\\
        F(n_{1}n_{2}).\\
    \end{bmatrix}
\end{equation}
\end{homeworkProblem}
\begin{homeworkProblem}
(Page 245,Problem 5.1.8)

First, I derive that $Q$ is an orthogonal matrix.
\begin{equation}
    \begin{aligned}
    QQ^{t}&=(I+S)(I-S)^{-1}(I-S^{T})^{-1}(I+S^{T})\\
    &=(I+S)(I-S)^{-1}(I+S)^{-1}(I-S)\\
    &=(I+S)(I+S)^{-1}(I-S)^{-1}(I-S)\\
    &=I.
    \end{aligned}
\end{equation}
It means that $Q$ is an orthogonal matrix.

Assume the Householder transformation from $x$ to $\|x\|e_{1}$ is 
\begin{equation}
    H=I-\beta vv^{t},
\end{equation}
then:
\begin{equation}
    \begin{aligned}
        (I+S)(I-S)^{-1}&=I-\beta vv^{t}\\
        (I+S)&=(I-S)(I-\beta vv^{t})\\
        S(2I-\beta vv^{t})&=-\beta vv^{t}\\
        S&=-\beta(2I-\beta vv^{t})^{-1}vv^{t}.
    \end{aligned}
\end{equation}
This is exactly the rank-2 anti-symmetric matrix $S$.
\end{homeworkProblem}
\end{document}