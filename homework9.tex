\documentclass{article}

\usepackage{fancyhdr}
\usepackage{extramarks}
\usepackage{amsmath}
\usepackage{amsthm}
\usepackage{amsfonts}
\usepackage{tikz}
\usepackage[plain]{algorithm}
\usepackage{algpseudocode}

\usetikzlibrary{automata,positioning}

%
% Basic Document Settings
%

\topmargin=-0.45in
\evensidemargin=0in
\oddsidemargin=0in
\textwidth=6.5in
\textheight=9.0in
\headsep=0.25in

\linespread{1.1}

\pagestyle{fancy}
\lhead{\hmwkAuthorName}
\chead{\hmwkClass\ (\hmwkClassInstructor\ \hmwkClassTime): \hmwkTitle}
\rhead{\firstxmark}
\lfoot{\lastxmark}
\cfoot{\thepage}

\renewcommand\headrulewidth{0.4pt}
\renewcommand\footrulewidth{0.4pt}

\setlength\parindent{0pt}

%
% Create Problem Sections
%

\newcommand{\enterProblemHeader}[1]{
    \nobreak\extramarks{}{Problem \arabic{#1} continued on next page\ldots}\nobreak{}
    \nobreak\extramarks{Problem \arabic{#1} (continued)}{Problem \arabic{#1} continued on next page\ldots}\nobreak{}
}

\newcommand{\exitProblemHeader}[1]{
    \nobreak\extramarks{Problem \arabic{#1} (continued)}{Problem \arabic{#1} continued on next page\ldots}\nobreak{}
    \stepcounter{#1}
    \nobreak\extramarks{Problem \arabic{#1}}{}\nobreak{}
}

\setcounter{secnumdepth}{0}
\newcounter{partCounter}
\newcounter{homeworkProblemCounter}
\setcounter{homeworkProblemCounter}{1}
\nobreak\extramarks{Problem \arabic{homeworkProblemCounter}}{}\nobreak{}

%
% Homework Problem Environment
%
% This environment takes an optional argument. When given, it will adjust the
% problem counter. This is useful for when the problems given for your
% assignment aren't sequential. See the last 3 problems of this template for an
% example.
%
\newenvironment{homeworkProblem}[1][-1]{
    \ifnum#1>0
        \setcounter{homeworkProblemCounter}{#1}
    \fi
    \section{Problem \arabic{homeworkProblemCounter}}
    \setcounter{partCounter}{1}
    \enterProblemHeader{homeworkProblemCounter}
}{
    \exitProblemHeader{homeworkProblemCounter}
}

%
% Homework Details
%   - Title
%   - Due date
%   - Class
%   - Section/Time
%   - Instructor
%   - Author
%

\newcommand{\hmwkTitle}{Homework\ \#9}
\newcommand{\hmwkDueDate}{Nov 21, 2022}
\newcommand{\hmwkClass}{Matrix Calculation}
\newcommand{\hmwkClassTime}{Monday}
\newcommand{\hmwkClassInstructor}{Professor Jun Lai}
\newcommand{\hmwkAuthorName}{\textbf{Shuang Hu}}

%
% Title Page
%

\title{
    \vspace{2in}
    \textmd{\textbf{\hmwkClass:\ \hmwkTitle}}\\
    \normalsize\vspace{0.1in}\small{Due\ on\ \hmwkDueDate\ at 3:10pm}\\
    \vspace{0.1in}\large{\textit{\hmwkClassInstructor\ \hmwkClassTime}}
    \vspace{3in}
}

\author{\hmwkAuthorName}
\date{}

\renewcommand{\part}[1]{\textbf{\large Part \Alph{partCounter}}\stepcounter{partCounter}\\}

%
% Various Helper Commands
%

% Useful for algorithms
\newcommand{\alg}[1]{\textsc{\bfseries \footnotesize #1}}

% For derivatives
\newcommand{\deriv}[1]{\frac{\mathrm{d}}{\mathrm{d}x} (#1)}

% For partial derivatives
\newcommand{\pderiv}[2]{\frac{\partial}{\partial #1} (#2)}

% Integral dx
\newcommand{\dx}{\mathrm{d}x}

% Alias for the Solution section header
\newcommand{\solution}{\textbf{\large Solution}}
\newcommand{\norm}[1]{\|#1\|}
% Probability commands: Expectation, Variance, Covariance, Bias
\newcommand{\E}{\mathrm{E}}
\newcommand{\Var}{\mathrm{Var}}
\newcommand{\Cov}{\mathrm{Cov}}
\newcommand{\Bias}{\mathrm{Bias}}
\newcommand{\supp}{\text{supp}}
\newcommand{\Rn}{\mathbb{R}^{n}}
\newcommand{\dif}{\mathrm{d}}
\newcommand{\avg}[1]{\left\langle #1 \right\rangle}
\newcommand{\difFrac}[2]{\frac{\dif #1}{\dif #2}}
\newcommand{\pdfFrac}[2]{\frac{\partial #1}{\partial #2}}
\newcommand{\OFL}{\mathrm{OFL}}
\newcommand{\UFL}{\mathrm{UFL}}
\newcommand{\fl}{\mathrm{fl}}
\newcommand{\op}{\odot}
\newcommand{\cp}{\cdot}
\newcommand{\Eabs}{E_{\mathrm{abs}}}
\newcommand{\Erel}{E_{\mathrm{rel}}}
\newcommand{\DR}{\mathcal{D}_{\widetilde{LN}}^{n}}
\newcommand{\add}[2]{\sum_{#1=1}^{#2}}
\newcommand{\innerprod}[2]{\left<#1,#2\right>}
\newcommand\tbbint{{-\mkern -16mu\int}}
\newcommand\tbint{{\mathchar '26\mkern -14mu\int}}
\newcommand\dbbint{{-\mkern -19mu\int}}
\newcommand\dbint{{\mathchar '26\mkern -18mu\int}}
\newcommand\bint{
{\mathchoice{\dbint}{\tbint}{\tbint}{\tbint}}
}
\newcommand\bbint{
{\mathchoice{\dbbint}{\tbbint}{\tbbint}{\tbbint}}
}
\begin{document}
\maketitle
\pagebreak
\begin{homeworkProblem}
(Page 296,Problem 5.5.1)

By the definition, 
\begin{equation}
    A=\begin{bmatrix}
        T&S\\
        O&O\\
    \end{bmatrix}, X=\begin{bmatrix}
        T^{-1}&O\\
        O&O\\
    \end{bmatrix}.
\end{equation}
We can see that 
\begin{equation}
    AX=\begin{bmatrix}
        I&O\\
        O&O\\
    \end{bmatrix}=(AX)^{T},
\end{equation}
and 
\begin{equation}
    AXA=\begin{bmatrix}
        I&O\\
        O&O\\
    \end{bmatrix}A=A.
\end{equation}
So $X$ is a (1,3) pseudoinverse of $A$.

By the definition, if $x_{B}=Xb$, we can see that 
\begin{equation}
    x_{B}=\Pi\begin{bmatrix}
        R_{11}^{-1}c\\
        O\\
    \end{bmatrix}
    =\Pi\begin{bmatrix}
        R_{11}^{-1}&O\\
        O&O\\
    \end{bmatrix}
    Q^{T}b.
\end{equation}
So 
\begin{equation}
    X=\Pi\begin{bmatrix}
        R_{11}^{-1}&O\\
        O&O\\
    \end{bmatrix}Q^{T}.
\end{equation}
Then:
\begin{equation}
    AX=A\Pi\begin{bmatrix}
        R_{11}^{-1}&O\\
        O&O\\
    \end{bmatrix}
    Q^{T}=Q\begin{bmatrix}
        R_{11}&R_{12}\\
        O&O\\
    \end{bmatrix}
    \begin{bmatrix}
        R_{11}^{-1}&O\\
        O&O\\
    \end{bmatrix}Q^{T}
    =\begin{bmatrix}
        I&O\\
        O&O\\
    \end{bmatrix}=(AX)^{T}.
\end{equation}
and 
\begin{equation}
    AXA=A.
\end{equation}
It means that $X$ is the (1,3) pseudoinverse of $A$.
\end{homeworkProblem}
\begin{homeworkProblem}
    By SVD, we can see that 
    \begin{equation}
        A=U\Sigma V^{T}, A^{+}=V\Sigma^{+}U^{T}.
    \end{equation}
    Then:
    \begin{equation}
        B(\lambda)=(A^{T}A+\lambda I)^{-1}A^{T}=V(\Sigma^{T}\Sigma+\lambda I)^{-1}V^{T}A^{T}.=V(\Sigma^{T}\Sigma+\lambda I)^{-1}\Sigma U^{T}.
    \end{equation}
    So 
    \begin{equation}
        B(\lambda)-A^{+}=V\left[(\Sigma^{T}\Sigma+\lambda I)^{-1}\Sigma-\Sigma^{+}\right]U^{T}.
    \end{equation}
    It means that 
    \begin{equation}
        \norm{B(\lambda)-A^{+}}_{2}=\norm{(\Sigma^{T}\Sigma+\lambda I)^{-1}\Sigma-\Sigma^{+}}_{2}.
    \end{equation}
    Consider the matrix righthand, on the $i$-th position, we can see 
    \begin{equation}
        a_{ii}=\frac{\sigma_{i}}{\sigma_{i}^{2}+\lambda}-\frac{1}{\sigma_{i}}=\frac{-\lambda}{\sigma_{i}(\sigma_{i}^{2}+\lambda)}, i\le r.
        a_{ii}=0, i>r.
        a_{ij}=0, i\neq j.
    \end{equation}
    So 
    \begin{equation}
        \norm{B(\lambda)-A^{+}}_{2}=\frac{\lambda}{\sigma_{r}(\sigma_{r}^{2}+\lambda)}.
    \end{equation}
\end{homeworkProblem}
\begin{homeworkProblem}
(Page 301, P5.6.2)

\begin{equation}
    Ax=b\Rightarrow x=\lambda_{1}e_{1}+\frac{1}{2}\lambda_{2}e_{2}+\frac{1}{3}\lambda_{3}e_{3}.
\end{equation}
while $\lambda_{1}+\lambda_{2}+\lambda_{3}=1$. To minimize $\norm{x}$, just set $\lambda_{3}=1$ and $\lambda_{1}=\lambda_{2}=0$, then $x=\frac{1}{3}e_{3}$ is the minimal norm solution.
\end{homeworkProblem}
\end{document}