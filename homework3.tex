\documentclass{article}

\usepackage{fancyhdr}
\usepackage{extramarks}
\usepackage{amsmath}
\usepackage{amsthm}
\usepackage{amsfonts}
\usepackage{tikz}
\usepackage[plain]{algorithm}
\usepackage{algpseudocode}

\usetikzlibrary{automata,positioning}

%
% Basic Document Settings
%

\topmargin=-0.45in
\evensidemargin=0in
\oddsidemargin=0in
\textwidth=6.5in
\textheight=9.0in
\headsep=0.25in

\linespread{1.1}

\pagestyle{fancy}
\lhead{\hmwkAuthorName}
\chead{\hmwkClass\ (\hmwkClassInstructor\ \hmwkClassTime): \hmwkTitle}
\rhead{\firstxmark}
\lfoot{\lastxmark}
\cfoot{\thepage}

\renewcommand\headrulewidth{0.4pt}
\renewcommand\footrulewidth{0.4pt}

\setlength\parindent{0pt}

%
% Create Problem Sections
%

\newcommand{\enterProblemHeader}[1]{
    \nobreak\extramarks{}{Problem \arabic{#1} continued on next page\ldots}\nobreak{}
    \nobreak\extramarks{Problem \arabic{#1} (continued)}{Problem \arabic{#1} continued on next page\ldots}\nobreak{}
}

\newcommand{\exitProblemHeader}[1]{
    \nobreak\extramarks{Problem \arabic{#1} (continued)}{Problem \arabic{#1} continued on next page\ldots}\nobreak{}
    \stepcounter{#1}
    \nobreak\extramarks{Problem \arabic{#1}}{}\nobreak{}
}

\setcounter{secnumdepth}{0}
\newcounter{partCounter}
\newcounter{homeworkProblemCounter}
\setcounter{homeworkProblemCounter}{1}
\nobreak\extramarks{Problem \arabic{homeworkProblemCounter}}{}\nobreak{}

%
% Homework Problem Environment
%
% This environment takes an optional argument. When given, it will adjust the
% problem counter. This is useful for when the problems given for your
% assignment aren't sequential. See the last 3 problems of this template for an
% example.
%
\newenvironment{homeworkProblem}[1][-1]{
    \ifnum#1>0
        \setcounter{homeworkProblemCounter}{#1}
    \fi
    \section{Problem \arabic{homeworkProblemCounter}}
    \setcounter{partCounter}{1}
    \enterProblemHeader{homeworkProblemCounter}
}{
    \exitProblemHeader{homeworkProblemCounter}
}

%
% Homework Details
%   - Title
%   - Due date
%   - Class
%   - Section/Time
%   - Instructor
%   - Author
%

\newcommand{\hmwkTitle}{Homework\ \#3}
\newcommand{\hmwkDueDate}{Oct 10, 2022}
\newcommand{\hmwkClass}{Matrix Calculation}
\newcommand{\hmwkClassTime}{Monday}
\newcommand{\hmwkClassInstructor}{Professor Jun Lai}
\newcommand{\hmwkAuthorName}{\textbf{Shuang Hu}}

%
% Title Page
%

\title{
    \vspace{2in}
    \textmd{\textbf{\hmwkClass:\ \hmwkTitle}}\\
    \normalsize\vspace{0.1in}\small{Due\ on\ \hmwkDueDate\ at 3:10pm}\\
    \vspace{0.1in}\large{\textit{\hmwkClassInstructor\ \hmwkClassTime}}
    \vspace{3in}
}

\author{\hmwkAuthorName}
\date{}

\renewcommand{\part}[1]{\textbf{\large Part \Alph{partCounter}}\stepcounter{partCounter}\\}

%
% Various Helper Commands
%

% Useful for algorithms
\newcommand{\alg}[1]{\textsc{\bfseries \footnotesize #1}}

% For derivatives
\newcommand{\deriv}[1]{\frac{\mathrm{d}}{\mathrm{d}x} (#1)}

% For partial derivatives
\newcommand{\pderiv}[2]{\frac{\partial}{\partial #1} (#2)}

% Integral dx
\newcommand{\dx}{\mathrm{d}x}

% Alias for the Solution section header
\newcommand{\solution}{\textbf{\large Solution}}

% Probability commands: Expectation, Variance, Covariance, Bias
\newcommand{\E}{\mathrm{E}}
\newcommand{\Var}{\mathrm{Var}}
\newcommand{\Cov}{\mathrm{Cov}}
\newcommand{\Bias}{\mathrm{Bias}}
\newcommand{\supp}{\text{supp}}
\newcommand{\Rn}{\mathbb{R}^{n}}
\newcommand{\dif}{\mathrm{d}}
\newcommand{\avg}[1]{\left\langle #1 \right\rangle}
\newcommand{\difFrac}[2]{\frac{\dif #1}{\dif #2}}
\newcommand{\pdfFrac}[2]{\frac{\partial #1}{\partial #2}}
\newcommand{\OFL}{\mathrm{OFL}}
\newcommand{\UFL}{\mathrm{UFL}}
\newcommand{\fl}{\mathrm{fl}}
\newcommand{\op}{\odot}
\newcommand{\cp}{\cdot}
\newcommand{\Eabs}{E_{\mathrm{abs}}}
\newcommand{\Erel}{E_{\mathrm{rel}}}
\newcommand{\DR}{\mathcal{D}_{\widetilde{LN}}^{n}}
\newcommand{\add}[2]{\sum_{#1=1}^{#2}}
\newcommand{\innerprod}[2]{\left<#1,#2\right>}
\newcommand\tbbint{{-\mkern -16mu\int}}
\newcommand\tbint{{\mathchar '26\mkern -14mu\int}}
\newcommand\dbbint{{-\mkern -19mu\int}}
\newcommand\dbint{{\mathchar '26\mkern -18mu\int}}
\newcommand\bint{
{\mathchoice{\dbint}{\tbint}{\tbint}{\tbint}}
}
\newcommand\bbint{
{\mathchoice{\dbbint}{\tbbint}{\tbbint}{\tbbint}}
}
\begin{document}
\maketitle
\pagebreak
\begin{homeworkProblem}

(Page 92, Problem 2.6.5)

\begin{proof}
    Set $u:=e_{i}$, $v:=\epsilon e_{j}$, we can see $A+\epsilon a_{ij}E_{ij}=A+uv^{T}$. Set $x=A^{-1}b$, $\tilde{x}=(A+\epsilon a_{ij})^{-1}b$, and $C:=A^{-1}$, we can see:
    \begin{equation}
        \tilde{x}=(A+uv^{t})^{-1}b=(C-\frac{Cuv^{t}C}{1+v^{t}Cu})b.
    \end{equation}

    Then:
    \begin{equation}
        x-\tilde{x}=\frac{Cuv^{t}Cb}{1+v^{t}Cu}=\frac{Cuv^{t}x}{1+\epsilon c_{ji}}=\frac{\epsilon x_{j}Ce_{i}}{1+\epsilon c_{ji}}=\frac{\epsilon x_{j}}{1+\epsilon c_{ji}}\begin{bmatrix}
            c_{1i}\\
            c_{2i}\\
            \vdots\\
            c_{ni}\\
        \end{bmatrix}.
    \end{equation}

    We can see:
    \begin{equation}
        \frac{\tilde{x}_{k}-x_{k}}{\epsilon}=-\frac{x_{j}}{1+\epsilon c_{ji}}c_{ki}\rightarrow -x_{j}c_{ki}.
    \end{equation}
\end{proof}
\end{homeworkProblem}
\begin{homeworkProblem}
    
(Page 120, Problem 3.2.2)
\begin{proof}
For matrix $A(\epsilon)=(a_{ij}(\epsilon))_{i,j=1}^{n}$, as the problem suggests, $a_{ij}(\epsilon)$ must be $C^{1}$ function. Then, for the principal matrices $A_{k}(\epsilon)=A(1:k,1:k)$, $\det(A_{k}(\epsilon))$ must be $C^{1}$ as well. And $\det(A_{k}(0))\neq 0$, so we can choose $\epsilon_{k}\in (0,w_{k})$ such that $\det(A_{k}(\epsilon_{k}))\neq 0$, where $w_{k}>0$. Then set $w=\min_{k=1}^{N}w_{k}$, we can see $w>0$ and $\forall\epsilon\in(0,w)$, $\det(A_{k}(\epsilon))\neq 0$. This means that $A(\epsilon)$ has an LU factorization.

Then, using induction to prove the property of the LU factorization. For $n=1$, we can see $L=(1)$ and $U=(a_{11}(\epsilon))$, $L$ and $U$ are both $C^{1}$. Assume this factorization exists for $n=k$, consider $n=k+1$, in this condition, 
\begin{equation}
    A(\epsilon)=\begin{bmatrix}
        A_{k}(\epsilon)&\beta(\epsilon)\\
        \alpha^{t}(\epsilon)&\varphi(\epsilon)
    \end{bmatrix}.
\end{equation}
By induction, $A_{k}(\epsilon)=L_{k}(\epsilon)U_{k}(\epsilon)$ where $L_{k}, U_{k}$ both non-regular.

Then, set 
\begin{equation}
    L(\epsilon)=\begin{bmatrix}
        L_{k}(\epsilon)&0\\
        U_{k}^{-1}(\epsilon)\alpha^{t}(\epsilon)&1\\
    \end{bmatrix},
    U(\epsilon)=\begin{bmatrix}
        U_{k}(\epsilon)&L_{k}(\epsilon)^{-1}\beta(\epsilon)\\
        0&\varphi(\epsilon)-\alpha(\epsilon)(U_{k}^{-1}(\epsilon))^{t}L_{k}^{-1}(\epsilon)\beta(\epsilon)\\
    \end{bmatrix}.
\end{equation}
We can see $L(\epsilon)U(\epsilon)=A(\epsilon)$, and they are both $C^{1}$.
\end{proof}
\end{homeworkProblem}
\end{document}