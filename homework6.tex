\documentclass{article}

\usepackage{fancyhdr}
\usepackage{extramarks}
\usepackage{amsmath}
\usepackage{amsthm}
\usepackage{amsfonts}
\usepackage{tikz}
\usepackage[plain]{algorithm}
\usepackage{algpseudocode}

\usetikzlibrary{automata,positioning}

%
% Basic Document Settings
%

\topmargin=-0.45in
\evensidemargin=0in
\oddsidemargin=0in
\textwidth=6.5in
\textheight=9.0in
\headsep=0.25in

\linespread{1.1}

\pagestyle{fancy}
\lhead{\hmwkAuthorName}
\chead{\hmwkClass\ (\hmwkClassInstructor\ \hmwkClassTime): \hmwkTitle}
\rhead{\firstxmark}
\lfoot{\lastxmark}
\cfoot{\thepage}

\renewcommand\headrulewidth{0.4pt}
\renewcommand\footrulewidth{0.4pt}

\setlength\parindent{0pt}

%
% Create Problem Sections
%

\newcommand{\enterProblemHeader}[1]{
    \nobreak\extramarks{}{Problem \arabic{#1} continued on next page\ldots}\nobreak{}
    \nobreak\extramarks{Problem \arabic{#1} (continued)}{Problem \arabic{#1} continued on next page\ldots}\nobreak{}
}

\newcommand{\exitProblemHeader}[1]{
    \nobreak\extramarks{Problem \arabic{#1} (continued)}{Problem \arabic{#1} continued on next page\ldots}\nobreak{}
    \stepcounter{#1}
    \nobreak\extramarks{Problem \arabic{#1}}{}\nobreak{}
}

\setcounter{secnumdepth}{0}
\newcounter{partCounter}
\newcounter{homeworkProblemCounter}
\setcounter{homeworkProblemCounter}{1}
\nobreak\extramarks{Problem \arabic{homeworkProblemCounter}}{}\nobreak{}

%
% Homework Problem Environment
%
% This environment takes an optional argument. When given, it will adjust the
% problem counter. This is useful for when the problems given for your
% assignment aren't sequential. See the last 3 problems of this template for an
% example.
%
\newenvironment{homeworkProblem}[1][-1]{
    \ifnum#1>0
        \setcounter{homeworkProblemCounter}{#1}
    \fi
    \section{Problem \arabic{homeworkProblemCounter}}
    \setcounter{partCounter}{1}
    \enterProblemHeader{homeworkProblemCounter}
}{
    \exitProblemHeader{homeworkProblemCounter}
}

%
% Homework Details
%   - Title
%   - Due date
%   - Class
%   - Section/Time
%   - Instructor
%   - Author
%

\newcommand{\hmwkTitle}{Homework\ \#6}
\newcommand{\hmwkDueDate}{Oct 31, 2022}
\newcommand{\hmwkClass}{Matrix Calculation}
\newcommand{\hmwkClassTime}{Monday}
\newcommand{\hmwkClassInstructor}{Professor Jun Lai}
\newcommand{\hmwkAuthorName}{\textbf{Shuang Hu}}

%
% Title Page
%

\title{
    \vspace{2in}
    \textmd{\textbf{\hmwkClass:\ \hmwkTitle}}\\
    \normalsize\vspace{0.1in}\small{Due\ on\ \hmwkDueDate\ at 3:10pm}\\
    \vspace{0.1in}\large{\textit{\hmwkClassInstructor\ \hmwkClassTime}}
    \vspace{3in}
}

\author{\hmwkAuthorName}
\date{}

\renewcommand{\part}[1]{\textbf{\large Part \Alph{partCounter}}\stepcounter{partCounter}\\}

%
% Various Helper Commands
%

% Useful for algorithms
\newcommand{\alg}[1]{\textsc{\bfseries \footnotesize #1}}

% For derivatives
\newcommand{\deriv}[1]{\frac{\mathrm{d}}{\mathrm{d}x} (#1)}

% For partial derivatives
\newcommand{\pderiv}[2]{\frac{\partial}{\partial #1} (#2)}

% Integral dx
\newcommand{\dx}{\mathrm{d}x}

% Alias for the Solution section header
\newcommand{\solution}{\textbf{\large Solution}}
\newcommand{\norm}[1]{\|#1\|}
% Probability commands: Expectation, Variance, Covariance, Bias
\newcommand{\E}{\mathrm{E}}
\newcommand{\Var}{\mathrm{Var}}
\newcommand{\Cov}{\mathrm{Cov}}
\newcommand{\Bias}{\mathrm{Bias}}
\newcommand{\supp}{\text{supp}}
\newcommand{\Rn}{\mathbb{R}^{n}}
\newcommand{\dif}{\mathrm{d}}
\newcommand{\avg}[1]{\left\langle #1 \right\rangle}
\newcommand{\difFrac}[2]{\frac{\dif #1}{\dif #2}}
\newcommand{\pdfFrac}[2]{\frac{\partial #1}{\partial #2}}
\newcommand{\OFL}{\mathrm{OFL}}
\newcommand{\UFL}{\mathrm{UFL}}
\newcommand{\fl}{\mathrm{fl}}
\newcommand{\op}{\odot}
\newcommand{\cp}{\cdot}
\newcommand{\Eabs}{E_{\mathrm{abs}}}
\newcommand{\Erel}{E_{\mathrm{rel}}}
\newcommand{\DR}{\mathcal{D}_{\widetilde{LN}}^{n}}
\newcommand{\add}[2]{\sum_{#1=1}^{#2}}
\newcommand{\innerprod}[2]{\left<#1,#2\right>}
\newcommand\tbbint{{-\mkern -16mu\int}}
\newcommand\tbint{{\mathchar '26\mkern -14mu\int}}
\newcommand\dbbint{{-\mkern -19mu\int}}
\newcommand\dbint{{\mathchar '26\mkern -18mu\int}}
\newcommand\bint{
{\mathchoice{\dbint}{\tbint}{\tbint}{\tbint}}
}
\newcommand\bbint{
{\mathchoice{\dbbint}{\tbbint}{\tbbint}{\tbbint}}
}
\begin{document}
\maketitle
\pagebreak
\begin{homeworkProblem}
    (Page41 Problem1.4.2)

    Set 
    \begin{equation}
        G=\begin{bmatrix}
            F_{11}&F_{12}&F_{13}\\
            F_{21}&F_{22}&F_{23}\\
            F_{31}&F_{32}&F_{33}\\
        \end{bmatrix}
    \end{equation}
    By the definition of $F_{n}(:,1:3:n-1)$, we can see that:
    \begin{equation}
        F_{11}=F_{21}=F_{31}=\begin{bmatrix}
            1&1&\cdots&1\\
            1&\omega_{3m}^{3}&\cdots&\omega_{3m}^{3(m-1)}\\
            \vdots&\vdots&\ddots&\vdots\\
            1&\omega_{3m}^{3(m-1)}&\cdots&\omega_{3m}^{3(m-1)^{2}}\\
        \end{bmatrix}
        =\begin{bmatrix}
            1&1&\cdots&1\\
            1&\omega_{m}&\cdots&\omega_{m}^{m-1}\\
            \vdots&\vdots&\ddots&\vdots\\
            1&\omega_{m}^{m-1}&\cdots&\omega_{m}^{(m-1)^{2}}\\
        \end{bmatrix}=F_{m}.
    \end{equation}
    Consider $F_{12}$, we can see:
    \begin{equation}
        F_{12}=\begin{bmatrix}
            1&1&\cdots&1\\
            \omega_{3m}&\omega_{3m}^{4}&\cdots&\omega_{3m}^{3m-2}\\
            \vdots&\vdots&\ddots&\vdots\\
            \omega_{3m}^{m-1}&\omega_{3m}^{4(m-1)}&\cdots&\omega_{3m}^{(3m-2)(m-1)}\\
        \end{bmatrix}
    \end{equation}
    which means that
    \begin{equation}
        F_{12}=\Omega_{m}F_{m},\Omega_{m}=\text{diag}(1,\omega_{3m},\cdots,\omega_{3m}^{m-1}).
    \end{equation}
    Similarly, we can write:
    \begin{equation}
        F_{22}=\omega_{3}\Omega_{m}F_{m}, F_{32}=\omega_{3}^{2}\Omega_{m}F_{m}.
    \end{equation}
    and:
    \begin{equation}
        F_{31}=\Omega_{m}^{2}F_{m}, F_{32}=\omega_{3}^{2}\Omega_{m}^{2}F_{m}, F_{33}=\omega_{3}^{4}\Omega_{m}^{2}F_{m}.
    \end{equation}
    Set $y_{a}=x(1:3:n-1),y_{b}=x(2:3:n-1),y_{c}=x(3:3:n-1)$, we can see:

    \begin{equation}
        F_{n}y=G(y_{a}^{t},y_{b}^{t},y_{c}^{t})^{t}=\begin{bmatrix}
            F_{m}y_{a}+\Omega_{m}F_{m}y_{b}+\Omega_{m}^{2}F_{m}y_{c}\\
            F_{m}y_{a}+\omega_{3}\Omega_{m}F_{m}y_{b}+\omega_{3}^{2}\Omega_{m}^{2}F_{m}y_{c}\\
            F_{m}y_{a}+\omega_{3}^{2}\Omega_{m}F_{m}y_{b}+\omega_{3}^{4}\Omega_{m}^{2}F_{m}y_{c}\\
        \end{bmatrix}
    \end{equation}
    Then we can derive an induction algorithm by equation $(7)$.
\end{homeworkProblem}
\end{document}