\documentclass{article}

\usepackage{fancyhdr}
\usepackage{extramarks}
\usepackage{amsmath}
\usepackage{amsthm}
\usepackage{amsfonts}
\usepackage{tikz}
\usepackage[plain]{algorithm}
\usepackage{algpseudocode}

\usetikzlibrary{automata,positioning}

%
% Basic Document Settings
%

\topmargin=-0.45in
\evensidemargin=0in
\oddsidemargin=0in
\textwidth=6.5in
\textheight=9.0in
\headsep=0.25in

\linespread{1.1}

\pagestyle{fancy}
\lhead{\hmwkAuthorName}
\chead{\hmwkClass\ (\hmwkClassInstructor\ \hmwkClassTime): \hmwkTitle}
\rhead{\firstxmark}
\lfoot{\lastxmark}
\cfoot{\thepage}

\renewcommand\headrulewidth{0.4pt}
\renewcommand\footrulewidth{0.4pt}

\setlength\parindent{0pt}

%
% Create Problem Sections
%

\newcommand{\enterProblemHeader}[1]{
    \nobreak\extramarks{}{Problem \arabic{#1} continued on next page\ldots}\nobreak{}
    \nobreak\extramarks{Problem \arabic{#1} (continued)}{Problem \arabic{#1} continued on next page\ldots}\nobreak{}
}

\newcommand{\exitProblemHeader}[1]{
    \nobreak\extramarks{Problem \arabic{#1} (continued)}{Problem \arabic{#1} continued on next page\ldots}\nobreak{}
    \stepcounter{#1}
    \nobreak\extramarks{Problem \arabic{#1}}{}\nobreak{}
}

\setcounter{secnumdepth}{0}
\newcounter{partCounter}
\newcounter{homeworkProblemCounter}
\setcounter{homeworkProblemCounter}{1}
\nobreak\extramarks{Problem \arabic{homeworkProblemCounter}}{}\nobreak{}

%
% Homework Problem Environment
%
% This environment takes an optional argument. When given, it will adjust the
% problem counter. This is useful for when the problems given for your
% assignment aren't sequential. See the last 3 problems of this template for an
% example.
%
\newenvironment{homeworkProblem}[1][-1]{
    \ifnum#1>0
        \setcounter{homeworkProblemCounter}{#1}
    \fi
    \section{Problem \arabic{homeworkProblemCounter}}
    \setcounter{partCounter}{1}
    \enterProblemHeader{homeworkProblemCounter}
}{
    \exitProblemHeader{homeworkProblemCounter}
}

%
% Homework Details
%   - Title
%   - Due date
%   - Class
%   - Section/Time
%   - Instructor
%   - Author
%

\newcommand{\hmwkTitle}{Homework\ \#1}
\newcommand{\hmwkDueDate}{Sept 15, 2022}
\newcommand{\hmwkClass}{Matrix Calculation}
\newcommand{\hmwkClassTime}{Monday}
\newcommand{\hmwkClassInstructor}{Professor Jun Lai}
\newcommand{\hmwkAuthorName}{\textbf{Shuang Hu}}

%
% Title Page
%

\title{
    \vspace{2in}
    \textmd{\textbf{\hmwkClass:\ \hmwkTitle}}\\
    \normalsize\vspace{0.1in}\small{Due\ on\ \hmwkDueDate\ at 3:10pm}\\
    \vspace{0.1in}\large{\textit{\hmwkClassInstructor\ \hmwkClassTime}}
    \vspace{3in}
}

\author{\hmwkAuthorName}
\date{}

\renewcommand{\part}[1]{\textbf{\large Part \Alph{partCounter}}\stepcounter{partCounter}\\}

%
% Various Helper Commands
%

% Useful for algorithms
\newcommand{\alg}[1]{\textsc{\bfseries \footnotesize #1}}

% For derivatives
\newcommand{\deriv}[1]{\frac{\mathrm{d}}{\mathrm{d}x} (#1)}

% For partial derivatives
\newcommand{\pderiv}[2]{\frac{\partial}{\partial #1} (#2)}

% Integral dx
\newcommand{\dx}{\mathrm{d}x}

% Alias for the Solution section header
\newcommand{\solution}{\textbf{\large Solution}}

% Probability commands: Expectation, Variance, Covariance, Bias
\newcommand{\E}{\mathrm{E}}
\newcommand{\Var}{\mathrm{Var}}
\newcommand{\Cov}{\mathrm{Cov}}
\newcommand{\Bias}{\mathrm{Bias}}
\newcommand{\supp}{\text{supp}}
\newcommand{\Rn}{\mathbb{R}^{n}}
\newcommand{\dif}{\mathrm{d}}
\newcommand{\avg}[1]{\left\langle #1 \right\rangle}
\newcommand{\difFrac}[2]{\frac{\dif #1}{\dif #2}}
\newcommand{\pdfFrac}[2]{\frac{\partial #1}{\partial #2}}
\newcommand{\OFL}{\mathrm{OFL}}
\newcommand{\UFL}{\mathrm{UFL}}
\newcommand{\fl}{\mathrm{fl}}
\newcommand{\op}{\odot}
\newcommand{\cp}{\cdot}
\newcommand{\Eabs}{E_{\mathrm{abs}}}
\newcommand{\Erel}{E_{\mathrm{rel}}}
\newcommand{\DR}{\mathcal{D}_{\widetilde{LN}}^{n}}
\newcommand{\add}[2]{\sum_{#1=1}^{#2}}
\newcommand{\innerprod}[2]{\left<#1,#2\right>}
\newcommand\tbbint{{-\mkern -16mu\int}}
\newcommand\tbint{{\mathchar '26\mkern -14mu\int}}
\newcommand\dbbint{{-\mkern -19mu\int}}
\newcommand\dbint{{\mathchar '26\mkern -18mu\int}}
\newcommand\bint{
{\mathchoice{\dbint}{\tbint}{\tbint}{\tbint}}
}
\newcommand\bbint{
{\mathchoice{\dbbint}{\tbbint}{\tbbint}{\tbbint}}
}

\begin{document}

\maketitle

\pagebreak

\begin{homeworkProblem}
    (Page 67, Problem 2.1.1)
\begin{proof}
    By the normalized standard form, as $r(A)=r$, we can write :
    \begin{equation}
        PAQ=\begin{bmatrix}
            I_{r}&O\\
            O&O\\
        \end{bmatrix}
    \end{equation}
    Then: 
    \begin{equation}
        \begin{aligned}
            A&=P^{-1}\begin{bmatrix}
                I_{r}&O\\
                O&O\\
            \end{bmatrix}Q^{-1}\\
            &=\left(P^{-1}\begin{bmatrix}
                I_{r}\\
                O\\
            \end{bmatrix}\right)\left(\begin{bmatrix}
                I_{r},O\\
            \end{bmatrix}Q^{-1}\right)
        \end{aligned}
    \end{equation}
    Set $X=P^{-1}\begin{bmatrix}
        I_{r}\\
        O\\
    \end{bmatrix}$, $Y^{t}=\begin{bmatrix}
        I_{r},O
    \end{bmatrix}Q^{-1}$, we can see $r(X)=r(Y)=r$, which satisfies the decomposition.
\end{proof}
    
\end{homeworkProblem}
\begin{homeworkProblem}
    (Page 67, Problem 2.1.2)

    \textbf{(a).}\begin{proof}
        \begin{equation}
            \begin{aligned}
            LHS&=\lim_{t\rightarrow 0}\frac{A(\alpha+t)B(\alpha+t)-A(\alpha)B(\alpha)}{t}\\
            &=\lim_{t\rightarrow 0}\frac{[A(\alpha+t)-A(\alpha)]B(\alpha+t)+A(\alpha)[B(\alpha+t)-B(\alpha)]}{t}\\
            &=\left[
                \difFrac{}{\alpha}A(\alpha)
            \right]B(\alpha)+A(\alpha)\left[\difFrac{}{\alpha}B(\alpha)\right]
            \end{aligned}
        \end{equation}
    \end{proof}

    \textbf{(b)}

    \begin{proof}
        As $I=A(\alpha)(A(\alpha)^{-1})$, $\difFrac{I}{\alpha}=0$ and \textbf{(a)} suggests, we can see:
        \begin{equation}
            \difFrac{I}{\alpha}=0=\difFrac{}{\alpha}\left[(A(\alpha))^{-1}\right]A(\alpha)+(A(\alpha))^{-1}\difFrac{}{\alpha}A(\alpha).
        \end{equation}
        Then:
        \begin{equation}
            \difFrac{}{\alpha}\left[A(\alpha)^{-1}\right]=-A(\alpha)^{-1}\left[\difFrac{}{\alpha}A(\alpha)\right]A(\alpha)^{-1}.
        \end{equation}
    \end{proof}
\end{homeworkProblem}
\begin{homeworkProblem}
    (Page 70, Problem 2.2.1)

    \begin{proof}
        In one hand:
        \begin{equation}
            \|\mathbf{x}\|_{p}=(\sum_{i=1}^{n}|x_{i}|^{p})^{\frac{1}{p}}\ge(\|\mathbf{x}\|_{\infty}^{p})^{\frac{1}{p}}=\|\mathbf{x}\|_{\infty}.
        \end{equation}
        In the other hand:
        \begin{equation}
            \|\mathbf{x}\|_{p}=(\sum_{i=1}^{n}|x_{i}|^{p})^{\frac{1}{p}}\le(n\|\mathbf{x}\|_{\infty}^{p})^{\frac{1}{p}}\rightarrow\|\mathbf{x}\|_{\infty}.
        \end{equation}
        Set $p\rightarrow\infty$, we can see: $\|\mathbf{x}\|_{p}\rightarrow\|\mathbf{x}\|_{\infty}$.
    \end{proof}
\end{homeworkProblem}
\begin{homeworkProblem}
    \begin{proof}
        On one hand:
        \begin{equation}
            \|x\|=\|x-y+y\|\le\|x-y\|+\|y\|\Rightarrow\|x\|-\|y\|\le\|x-y\|.
        \end{equation}
        On the other hand:
        \begin{equation}
            \|y\|=\|y-x+x\|\le\|x\|+\|y-x\|=\|x\|+\|x-y\|\Rightarrow\|y\|-\|x\|\le\|x-y\|.
        \end{equation}
        Then:
        \begin{equation}
            |\|x\|-\|y\||\le\|x-y\|.
        \end{equation}
    \end{proof}
\end{homeworkProblem}
\end{document}
